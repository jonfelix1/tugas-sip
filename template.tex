%  LaTeX support: latex@mdpi.com 
%  For support, please attach all files needed for compiling as well as the log file, and specify your operating system, LaTeX version, and LaTeX editor.

%=================================================================
\documentclass[journal,article,submit,pdftex,moreauthors]{Definitions/mdpi} 

%--------------------
% Class Options:
%--------------------
%----------
% journal
%----------
% Choose between the following MDPI journals:
% For posting an early version of this manuscript as a preprint, you may use "preprints" as the journal. Changing "submit" to "accept" before posting will remove line numbers.

%---------
% article
%---------
% The default type of manuscript is "article", but can be replaced by: 
% abstract, addendum, article, book, bookreview, briefreport, casereport, comment, commentary, communication, conferenceproceedings, correction, conferencereport, entry, expressionofconcern, extendedabstract, datadescriptor, editorial, essay, erratum, hypothesis, interestingimage, obituary, opinion, projectreport, reply, retraction, review, perspective, protocol, shortnote, studyprotocol, systematicreview, supfile, technicalnote, viewpoint, guidelines, registeredreport, tutorial
% supfile = supplementary materials

%----------
% submit
%----------
% The class option "submit" will be changed to "accept" by the Editorial Office when the paper is accepted. This will only make changes to the frontpage (e.g., the logo of the journal will get visible), the headings, and the copyright information. Also, line numbering will be removed. Journal info and pagination for accepted papers will also be assigned by the Editorial Office.

%------------------
% moreauthors
%------------------
% If there is only one author the class option oneauthor should be used. Otherwise use the class option moreauthors.

%---------
% pdftex
%---------
% The option pdftex is for use with pdfLaTeX. Remove "pdftex" for (1) compiling with LaTeX & dvi2pdf (if eps figures are used) or for (2) compiling with XeLaTeX.

%=================================================================
% MDPI internal commands - do not modify
\firstpage{1} 
\makeatletter 
\setcounter{page}{\@firstpage} 
\makeatother
\pubvolume{1}
\issuenum{1}
\articlenumber{0}
\pubyear{2024}
\copyrightyear{2024}
%\externaleditor{Academic Editor: Firstname Lastname}
\datereceived{ } 
\daterevised{ } % Comment out if no revised date
\dateaccepted{ } 
\datepublished{ } 
%\datecorrected{} % For corrected papers: "Corrected: XXX" date in the original paper.
%\dateretracted{} % For corrected papers: "Retracted: XXX" date in the original paper.
\hreflink{https://doi.org/} % If needed use \linebreak
%\doinum{}
%\pdfoutput=1 % Uncommented for upload to arXiv.org
%\CorrStatement{yes}  % For updates


%=================================================================
% Add packages and commands here. The following packages are loaded in our class file: fontenc, inputenc, calc, indentfirst, fancyhdr, graphicx, epstopdf, lastpage, ifthen, float, amsmath, amssymb, lineno, setspace, enumitem, mathpazo, booktabs, titlesec, etoolbox, tabto, xcolor, colortbl, soul, multirow, microtype, tikz, totcount, changepage, attrib, upgreek, array, tabularx, pbox, ragged2e, tocloft, marginnote, marginfix, enotez, amsthm, natbib, hyperref, cleveref, scrextend, url, geometry, newfloat, caption, draftwatermark, seqsplit
% cleveref: load \crefname definitions after \begin{document}

%=================================================================
% Please use the following mathematics environments: Theorem, Lemma, Corollary, Proposition, Characterization, Property, Problem, Example, ExamplesandDefinitions, Hypothesis, Remark, Definition, Notation, Assumption
%% For proofs, please use the proof environment (the amsthm package is loaded by the MDPI class).

%=================================================================
% Full title of the paper (Capitalized)
\Title{Analisis Faktor-Faktor Penentu Keberhasilan Implementasi Sistem Core Banking di Bank}

% MDPI internal command: Title for citation in the left column
\TitleCitation{Analisis Faktor-Faktor Penentu Keberhasilan Implementasi Sistem Core Banking di Bank}

% Author Orchid ID: enter ID or remove command
\newcommand{\orcidauthorA}{0000-0000-0000-000X} % Add \orcidA{} behind the author's name
%\newcommand{\orcidauthorB}{0000-0000-0000-000X} % Add \orcidB{} behind the author's name

% Authors, for the paper (add full first names)
% \Author{Firstname Lastname $^{1,\dagger,\ddagger}$\orcidA{}, Firstname Lastname $^{2,\ddagger}$ and Firstname Lastname $^{2,}$*}
\Author{Filzahanti Nuha Ramadhani$^{1,\ddagger}$, Samuel Mulatua Jeremy Nainggolan$^{1,\ddagger}$, Jon Felix Germinian$^{1,\ddagger}$}

%\longauthorlist{yes}

% MDPI internal command: Authors, for metadata in PDF
\AuthorNames{Firstname Lastname, Firstname Lastname and Firstname Lastname}

% MDPI internal command: Authors, for citation in the left column
\AuthorCitation{Lastname, F.; Lastname, F.; Lastname, F.}
% If this is a Chicago style journal: Lastname, Firstname, Firstname Lastname, and Firstname Lastname.

% Affiliations / Addresses (Add [1] after \address if there is only one affiliation.)
\address{%
$^{1}$ \quad Fakultas Ilmu Komputer Universitas Indonesia; filzahanti.nuha31@ui.ac.id\\
$^{2}$ \quad Fakultas Ilmu Komputer Universitas Indonesia; samuel.mulatua41@ui.ac.id\\
$^{3}$ \quad Fakultas Ilmu Komputer Universitas Indonesia; jon.felix@ui.ac.id\\
}

% Contact information of the corresponding author
\corres{Correspondence: e-mail@e-mail.com; Tel.: (optional; include country code; if there are multiple corresponding authors, add author initials) +xx-xxxx-xxx-xxxx (F.L.)}

% Current address and/or shared authorship
\firstnote{Current address: Affiliation.}  % Current address should not be the same as any items in the Affiliation section.
\secondnote{These authors contributed equally to this work.}
% The commands \thirdnote{} till \eighthnote{} are available for further notes

%\simplesumm{} % Simple summary

%\conference{} % An extended version of a conference paper

% Abstract (Do not insert blank lines, i.e. \\) 
\abstract{A single paragraph of about 200 words maximum. For research articles, abstracts should give a pertinent overview of the work. We strongly encourage authors to use the following style of structured abstracts, but without headings: (1) Background: place the question addressed in a broad context and highlight the purpose of the study; (2) Methods: describe briefly the main methods or treatments applied; (3) Results: summarize the article's main findings; (4) Conclusions: indicate the main conclusions or interpretations. The abstract should be an objective representation of the article, it must not contain results which are not presented and substantiated in the main text and should not exaggerate the main conclusions.}

% Keywords
\keyword{keyword 1; keyword 2; keyword 3 (List three to ten pertinent keywords specific to the article; yet reasonably common within the subject discipline.)} 

% The fields PACS, MSC, and JEL may be left empty or commented out if not applicable
%\PACS{J0101}
%\MSC{}
%\JEL{}

%%%%%%%%%%%%%%%%%%%%%%%%%%%%%%%%%%%%%%%%%%
% Only for the journal Diversity
%\LSID{\url{http://}}

%%%%%%%%%%%%%%%%%%%%%%%%%%%%%%%%%%%%%%%%%%
% Only for the journal Applied Sciences
%\featuredapplication{Authors are encouraged to provide a concise description of the specific application or a potential application of the work. This section is not mandatory.}
%%%%%%%%%%%%%%%%%%%%%%%%%%%%%%%%%%%%%%%%%%

%%%%%%%%%%%%%%%%%%%%%%%%%%%%%%%%%%%%%%%%%%
% Only for the journal Data
%\dataset{DOI number or link to the deposited data set if the data set is published separately. If the data set shall be published as a supplement to this paper, this field will be filled by the journal editors. In this case, please submit the data set as a supplement.}
%\datasetlicense{License under which the data set is made available (CC0, CC-BY, CC-BY-SA, CC-BY-NC, etc.)}

%%%%%%%%%%%%%%%%%%%%%%%%%%%%%%%%%%%%%%%%%%
% Only for the journal Toxins
%\keycontribution{The breakthroughs or highlights of the manuscript. Authors can write one or two sentences to describe the most important part of the paper.}

%%%%%%%%%%%%%%%%%%%%%%%%%%%%%%%%%%%%%%%%%%
% Only for the journal Encyclopedia
%\encyclopediadef{For entry manuscripts only: please provide a brief overview of the entry title instead of an abstract.}

%%%%%%%%%%%%%%%%%%%%%%%%%%%%%%%%%%%%%%%%%%
% Only for the journal Advances in Respiratory Medicine and Smart Cities
%\addhighlights{yes}
%\renewcommand{\addhighlights}{%

%\noindent This is an obligatory section in “Advances in Respiratory Medicine'' and ``Smart Cities”, whose goal is to increase the discoverability and readability of the article via search engines and other scholars. Highlights should not be a copy of the abstract, but a simple text allowing the reader to quickly and simplified find out what the article is about and what can be cited from it. Each of these parts should be devoted up to 2~bullet points.\vspace{3pt}\\
%\textbf{What are the main findings?}
% \begin{itemize}[labelsep=2.5mm,topsep=-3pt]
% \item First bullet.
% \item Second bullet.
% \end{itemize}\vspace{3pt}
%\textbf{What is the implication of the main finding?}
% \begin{itemize}[labelsep=2.5mm,topsep=-3pt]
% \item First bullet.
% \item Second bullet.
% \end{itemize}
%}

%%%%%%%%%%%%%%%%%%%%%%%%%%%%%%%%%%%%%%%%%%
\begin{document}

%%%%%%%%%%%%%%%%%%%%%%%%%%%%%%%%%%%%%%%%%%
% \setcounter{section}{-1} %% Remove this when starting to work on the template.
% \section{How to Use this Template}

% The template details the sections that can be used in a manuscript. Note that the order and names of article sections may differ from the requirements of the journal (e.g., the positioning of the Materials and Methods section). Please check the instructions on the authors' page of the journal to verify the correct order and names. For any questions, please contact the editorial office of the journal or support@mdpi.com. For LaTeX-related questions please contact latex@mdpi.com.%\endnote{This is an endnote.} % To use endnotes, please un-comment \printendnotes below (before References). Only journal Laws uses \footnote.

% The order of the section titles is different for some journals. Please refer to the "Instructions for Authors” on the journal homepage.

\section{Pendahuluan}

\subsection{Latar Belakang}

Dalam beberapa tahun terakhir, lembaga keuangan mikro seperti Bank Perekonomian Rakyat (BPR) dan Koperasi Simpan Pinjam (KSP) telah mengalami pertumbuhan pesat di Indonesia. Lembaga-lembaga ini memainkan peran penting dalam inklusi keuangan, terutama bagi masyarakat yang tidak memiliki akses ke perbankan konvensional. Namun, di tengah perkembangan ini, lembaga keuangan mikro menghadapi tantangan dalam mengelola operasional mereka, termasuk efisiensi pengelolaan data, keamanan transaksi, dan kecepatan pelayanan kepada nasabah.
Untuk mengatasi tantangan tersebut, adopsi teknologi informasi, khususnya aplikasi inti perbankan, atau yang kerap kali dikenal sebagai core banking system, menjadi solusi strategis. \textit{Core banking system} (CBS) memungkinkan lembaga keuangan mengelola berbagai aspek operasional secara terintegrasi—mulai dari manajemen simpanan, pinjaman, hingga pelaporan keuangan serta pelaporan regulasi pada regulator, seperti Bank Indonesia (BI), Otoritas Jasa Keuangan (OJK), dan lain-lain. Implementasi core banking system bertujuan untuk meningkatkan produktivitas, keamanan, dan pelayanan lembaga keuangan terhadap nasabah mereka.

PT Dimensi Kreasi Nusantara adalah salah satu penyedia solusi \textit{core banking system} yang menawarkan solusi teknologi informasi bagi lembaga keuangan mikro, mulai dari BPR hingga Koperasi Simpan Pinjam di seluruh Indonesia. Namun, implementasi teknologi ini tidaklah tanpa tantangan. Setiap lembaga keuangan mikro memiliki kebutuhan dan kondisi yang berbeda, tergantung pada ukuran aset, jumlah rekening yang dikelola, dan kompleksitas operasional.

Penelitian ini dilakukan untuk memahami lebih dalam mengenai faktor-faktor yang menentukan kesuksesan implementasi \textit{core banking system} pada lembaga keuangan mikro, dengan mengambil studi kasus klien-klien PT Dimensi Kreasi Nusantara dengan menganalisis aspek-aspek yang ada, baik aspek teknis maupun kesiapan organisasi, sumber daya manusia, dan proses bisnis yang ada, seperti dukungan manajemen, adaptasi teknologi oleh karyawan, serta ketersediaan infrastruktur. Selain itu, penelitian ini juga akan mengkaji perbedaan dalam keberhasilan implementasi di berbagai tipe lembaga keuangan mikro. Misalnya, bank dengan aset besar mungkin memiliki sumber daya lebih untuk mendukung implementasi teknologi dibandingkan dengan bank yang memiliki aset lebih kecil dan operasional yang lebih sederhana. Dengan demikian, pemahaman yang mendalam tentang berbagai faktor keberhasilan ini sangat penting untuk membantu lembaga keuangan mikro memaksimalkan manfaat dari \textit{core banking system}


\subsection{Problem Statement}

Penelitian ini bertujuan untuk mengeksplorasi faktor-faktor yang mempengaruhi keberhasilan implementasi \textit{core banking system} di lembaga keuangan mikro, menggunakan kerangka \textit{People}, \textit{Process}, dan \textit{Technology}. \textit{Core banking system} menjadi solusi strategis bagi lembaga keuangan mikro dalam meningkatkan efisiensi, keamanan, dan kecepatan pelayanan. Namun, adopsi teknologi ini menghadapi tantangan yang bervariasi tergantung pada ukuran aset, jumlah rekening, dan kompleksitas operasional. Oleh karena itu, penelitian ini akan mengkaji secara mendalam faktor-faktor yang berkontribusi terhadap keberhasilan implementasi sistem ini di berbagai kategori lembaga keuangan mikro

Dari perspektif \textit{People}, penelitian ini akan fokus pada bagaimana dukungan manajemen mempengaruhi keberhasilan implementasi \textit{core banking system}. Selain itu, aspek adaptasi teknologi oleh karyawan juga menjadi kunci penting yang akan diteliti, terutama dalam hal kesiapan sumber daya manusia dalam menerima perubahan teknologi. Perbedaan dalam tingkat kesiapan sumber daya manusia di lembaga keuangan mikro dengan aset besar dan kecil juga akan dianalisis untuk melihat dampaknya terhadap proses adopsi teknologi.

Dalam aspek \textit{Process}, penelitian ini akan mengeksplorasi bagaimana proses bisnis yang ada di lembaga keuangan mikro berdampak pada implementasi \textit{core banking system}. Kompleksitas operasional juga menjadi fokus, mengingat bahwa lembaga dengan struktur operasional yang lebih rumit mungkin menghadapi tantangan yang lebih besar dalam mengintegrasikan sistem baru. Penelitian ini juga akan mengeksplorasi bagaimana lembaga keuangan mikro mengelola perubahan proses kerja yang diperlukan untuk memaksimalkan manfaat dari \textit{core banking system}.

Dari sisi \textit{Technology}, penelitian ini akan mengidentifikasi tantangan yang berkaitan dengan infrastruktur teknologi yang digunakan oleh lembaga keuangan mikro. Ketersediaan teknologi yang memadai menjadi faktor penting dalam memastikan kelancaran implementasi dan operasional \textit{core banking system}. Selain itu, penelitian ini juga akan membandingkan kemampuan lembaga dengan kapasitas teknologi yang berbeda dalam mengadopsi dan memanfaatkan \textit{core banking system} secara efektif, terutama dalam konteks lembaga yang memiliki sumber daya teknologi terbatas

Dengan demikian pertanyaan yang diangkat di penelitian ini adalah sebagai berikut:
\begin{itemize}
    \item Apa saja faktor kunci yang menentukan keberhasilan implementasi \textit{core banking system} pada lembaga keuangan mikro?
    \item Bagaimana perbedaan faktor keberhasilan di antara berbagai kategori lembaga keuangan mikro berdasarkan aset, jumlah rekening, dan kompleksitas operasional?
    \item Apa tantangan utama yang dihadapi dalam implementasi \textit{core banking system} di lembaga keuangan mikro?
\end{itemize}

\subsection{Research Objectives}
Tujuan dari penelitian ini adalah untuk:
\begin{enumerate}
    \item Mengidentifikasi faktor kunci keberhasilan dalam implementasi core banking system di lembaga keuangan mikro.
    \item Menganalisis perbedaan faktor keberhasilan berdasarkan kategori lembaga keuangan mikro.
    \item Mengidentifikasi tantangan yang dihadapi oleh lembaga keuangan mikro dalam implementasi core banking system.
\end{enumerate}

\subsection{Manfaat Penelitian}
Penelitian ini dapat memberikan wawasan mengenai faktor-faktor kunci yang menentukan keberhasilan implementasi core banking system, sehingga dapat membantu lembaga keuangan mikro dalam merencanakan dan menjalankan proyek implementasi teknologi core banking system yang lebih efektif. Dengan memahami tantangan dan kebutuhan spesifik, lembaga keuangan dapat memaksimalkan efisiensi, produktivitas, dan pelayanan kepada nasabah.


% -------------------------------------------------------------------------------------------------------------------------------------------------


\section{Studi Literatrur} \label{sec:Studi Literatrur}

Pada bab ini menjelaskan tentang landasan teori yang digunakan dalam melakukan penelitian, yaitu mengenai \textit{core banking system} dan \textit{critical success factor}. Selain itu pada bab ini akan dibahas mengenai kajian dari hasil - hasil penelitian sebelumnya yang terkait dengan penelitian ini.

\subsection{Core Banking System} \label{sec:Core Banking System}

Core Banking System (CBS) adalah sistem informasi terpusat yang memungkinkan bank memproses dan mengelola berbagai transaksi finansial dan layanan perbankan secara real-time. CBS mencakup fungsi-fungsi dasar perbankan seperti transfer dana, pengelolaan akun, deposito, dan peminjaman. Sistem ini diakses oleh cabang-cabang bank dan platform digital secara langsung, sehingga menciptakan efisiensi dan konsistensi layanan perbankan. CBS merupakan "jantung" operasional bank yang mengotomatisasi sebagian besar proses internal dan eksternal untuk memastikan kelancaran transaksi harian serta pengelolaan data nasabah dengan aman dan cepat \cite{Hsiao-ebanking}.

\subsubsection{Manfaaat}

CBS memberikan banyak manfaat bagi institusi perbankan, termasuk efisiensi operasional yang lebih tinggi, integrasi data yang lebih baik, serta peningkatan pengalaman nasabah. Kualitas sistem dan informasi yang baik dapat meningkatkan kepuasan pengguna internal dan eksternal. Di PT Bank Perkreditan Rakyat (BPR) Hasamitra, kualitas sistem dan manfaat sistem CBS terbukti berpengaruh positif terhadap kinerja bank, termasuk penyediaan layanan yang lebih cepat dan andal \cite{basyir-cbs}. CBS juga memfasilitasi pengambilan keputusan yang lebih baik melalui integrasi data yang efektif, yang pada gilirannya meningkatkan daya saing bank di pasar \cite{pratama-cbs}.

\subsubsection{Tantangan Implementasi}

Meskipun CBS menawarkan berbagai manfaat, implementasi sistem ini dihadapkan pada sejumlah tantangan, terutama dalam hal adaptasi teknologi dan manajemen perubahan. Di Vietnam, penelitian menunjukkan bahwa kualitas layanan menjadi tantangan terbesar dalam memastikan kepuasan pengguna CBS, yang sering kali terhambat oleh ketergantungan pada infrastruktur teknologi yang ketinggalan zaman \cite{Hsiao-ebanking}. Selain itu, audit sistem CBS menggunakan ITIL menggarisbawahi perlunya pengelolaan siklus hidup layanan yang tepat untuk menghindari kegagalan operasional dan memastikan keberlanjutan sistem dalam jangka panjang \cite{wahyudi-cbs}. Implementasi CBS juga memerlukan penyesuaian budaya organisasi dan pelatihan intensif bagi karyawan untuk mengoptimalkan penggunaannya.

\subsection{Critical Success Factors}
    \textit{Critical Success Factors} (CSF) pada sistem informasi merujuk pada elemen-elemen kunci yang menentukan keberhasilan implementasi dan operasionalisasi suatu sistem informasi. CSF mencakup berbagai aspek yang harus diperhatikan secara cermat, termasuk faktor organisasi, teknis, dan pengguna. Dalam implementasi sistem informasi bisnis seperti \textit{Business Intelligence} (BI), CSF meliputi pemberdayaan organisasi, kemudahan penggunaan sistem, dan dukungan pengguna yang tepat \cite{harfoush2024critical}. Pada sistem DevOps, faktor kolaborasi, praktik teknis, dan pengukuran juga sangat krusial untuk memastikan kelancaran pengembangan perangkat lunak \cite{jayakody2023devops}. Sementara itu, dalam sistem informasi rumah sakit, CSF meliputi keandalan, kemudahan penggunaan, dan kecocokan organisasi yang berpengaruh besar terhadap keberhasilan sistem \cite{arpaci2023hospital}. Dalam sistem perbankan akan didiskusikan lebih lanjut pada subbab \ref{sec:Penelitian Terdahulu} 

\subsubsection{Manfaat Identifikasi Critical Success Factors (CSF)}
Mengidentifikasi dan memahami CSF dalam pengembangan sistem informasi memberikan berbagai manfaat penting. CSF membantu organisasi memastikan bahwa faktor-faktor utama yang mempengaruhi keberhasilan proyek sistem informasi dikelola dengan baik sejak awal. Dalam sistem informasi kesehatan, penerapan dan manajemen CSF secara prospektif dapat membantu dalam adaptasi situasional dan mengurangi risiko kegagalan sistem \cite{aggestam2023apply}. Lebih jauh lagi, pemahaman yang jelas tentang CSF memungkinkan organisasi untuk mengalokasikan sumber daya secara efektif, meningkatkan kepuasan pengguna, dan memastikan sistem tetap relevan dengan kebutuhan operasional dan teknologi yang berubah-ubah. Studi-studi menunjukkan bahwa pengelolaan CSF yang baik dapat mempercepat proses adopsi teknologi, meningkatkan produktivitas, dan mengurangi hambatan dalam pengembangan sistem informasi \cite{jayakody2023devops}.


\subsection{Penelitian Terdahulu} \label{sec:Penelitian Terdahulu}

Studi mengenai \textit{critical success factors} (CSFs) dalam implementasi sistem informasi \textit{Core Banking System} (CBS) menyoroti pentingnya dukungan manajemen puncak, metode manajemen proyek yang tepat, dan pelatihan pengguna akhir. Penelitian oleh Ghafari, H \cite{Ghafari-csf} menemukan bahwa faktor-faktor utama yang berpengaruh dalam keberhasilan implementasi CBS di Bank of Industry and Mine mencakup dukungan manajemen senior, rekayasa ulang proses bisnis, dan dukungan vendor. Dari faktor-faktor tersebut, dukungan manajemen senior dan pelatihan pengguna akhir memberikan dampak terbesar terhadap keberhasilan implementasi CBS.

Selain itu, penelitian di Nigeria menunjukkan bahwa manajemen risiko sistem informasi juga memainkan peran penting dalam keberhasilan CBS, terutama dalam mengatasi ancaman dan ketidakpastian yang terkait dengan informasi. Dukungan manajemen puncak, struktur organisasi, dan alokasi sumber daya yang memadai adalah kunci utama keberhasilan manajemen risiko sistem informasi di sektor perbankan Nigeria, yang pada akhirnya berkontribusi terhadap peningkatan kinerja bank \cite{falisat-csf}.

Studi lain \cite{salu-csf} di sektor perbankan India menyoroti sepuluh faktor sukses utama dalam adopsi CBS, seperti kompatibilitas, kompleksitas, dukungan manajemen, infrastruktur, keamanan, dan kebijakan pemerintah. Dengan menggunakan model struktural interpretatif, penelitian ini mengidentifikasi hubungan antara variabel-variabel tersebut dan menekankan pentingnya dukungan manajemen serta kesadaran akan keamanan sebagai faktor kunci adopsi CBS \cite{salu-csf}.

Kesuksesan implementasi CBS juga dipengaruhi oleh kesiapan organisasi dalam mengelola perubahan \cite{johny-csf}. Studi ini menemukan bahwa implementasi CBS yang efektif memerlukan pemahaman mendalam tentang proses implementasi, tantangan teknis, serta kesiapan teknologi dan tenaga kerja. Kurangnya pelatihan dan keterampilan teknis dapat menyebabkan kegagalan implementasi sistem CBS yang kompleks \cite{johny-csf}.



% -------------------------------------------------------------------------------------------------------------------------------------------------

\section{Metodologi}

% People
% - Dukungan manajemen senior \cite{Ghafari-csf} \cite{falisat-csf} 
% - Alokasi sumber daya \cite{falisat-csf} 
% - Kompatibilitas Teknologi \cite{salu-csf} \cite{johny-csf}
% - Perubahan manajemen dan Adaptasi User \cite{Ghafari-csf}

% Process
% - Rekayasa ulang proses bisnis \cite{falisat-csf}
% - Kebutuhan sistem yang jelas \cite{johny-csf}
% - Standar operasional yang jelas \cite{salu-csf}
% - Mitigasi and manajemen resiko \cite{falisat-csf}

% Teknologi
% - Keamanan yang kuat \cite{falisat-csf}
% - Skalabilitas infrastruktur IT \cite{Ghafari-csf} \cite{salu-csf}
% - Integrasi Sistem \cite{Ghafari-csf}
% - Performa dan kecepatan sistem \cite{Ghafari-csf}


\textit{Critical Success Factors} (CSF) sistem core banking diidentifikasi melalui tinjauan pustaka dari jurnal sistem informasi berkualitas tinggi yang diterbitkan antara tahun 2016 dan 2024 \cite{Ghafari-csf} \cite{falisat-csf} \cite{salu-csf} \cite{johny-csf}. Tinjauan ini difokuskan pada database seperti IEEE Xplore Digital Library, ScienceDirect, dan ACM Digital Library. Kata kunci yang digunakan untuk pencarian pustaka mencakup "\textit{critical success factors}", “\textit{success factors}”, "\textit{core banking system}", "\textit{banking system}", dan beberapa lainnya. Berdasarkan tinjauan pustaka yang telah dilakukan, didapat … faktor-faktor kesuksesan untuk digunakan dalam penelitian ini. CSF yang diteliti dapat dilihat pada tabel \ref{csf-table}



\begin{table}[H]
\begin{adjustwidth}{-\extralength}{0cm}
    \caption{Critical Success Factors (CSFs) dalam Implementasi Sistem Informasi}
    \label{csf-table}
    \centering
    \begin{tabular}{c|c|p{4cm}|p{6cm}|c}
        \toprule
        \textbf{Context} & \textbf{CSF Code} & \textbf{Critical Success Factors} & \textbf{Penjelasan Singkat} & \textbf{Reference} \\ 
        \midrule
        \textbf{People} & PP1 & \raggedright Dukungan manajemen senior & \raggedright Keterlibatan manajemen senior sangat penting dalam alokasi sumber daya dan arah proyek. & \cite{Ghafari-csf}, \cite{falisat-csf} \\ 
        \midrule
        \textbf{People} & PP2 & \raggedright Alokasi sumber daya & \raggedright Sumber daya yang memadai diperlukan untuk keberhasilan implementasi sistem. & \cite{falisat-csf} \\ 
        \midrule
        \textbf{People} & PP3 & \raggedright Kompatibilitas Teknologi & \raggedright Teknologi yang kompatibel dengan infrastruktur yang ada sangat penting untuk integrasi yang lancar. & \cite{salu-csf}, \cite{johny-csf} \\ 
        \midrule
        \textbf{People} & PP4 & \raggedright Perubahan manajemen dan Adaptasi User & \raggedright Manajemen perubahan yang baik membantu pengguna beradaptasi dengan sistem baru. & \cite{Ghafari-csf} \\ 
        \midrule
        \textbf{Process} & PR1 & \raggedright Rekayasa ulang proses bisnis & \raggedright Proses bisnis perlu direkayasa ulang untuk mendukung implementasi sistem yang lebih efisien. & \cite{falisat-csf} \\ 
        \midrule
        \textbf{Process} & PR2 & \raggedright Kebutuhan sistem yang jelas & \raggedright Kebutuhan sistem yang jelas membantu memastikan kesuksesan implementasi. & \cite{johny-csf} \\ 
        \midrule
        \textbf{Process} & PR3 & \raggedright Standar operasional yang jelas & \raggedright Standar operasional diperlukan untuk menjaga konsistensi dan efisiensi. & \cite{salu-csf} \\ 
        \midrule
        \textbf{Process} & PR4 & \raggedright Mitigasi dan manajemen risiko & \raggedright Penting untuk mengidentifikasi dan mengelola risiko untuk mencegah kegagalan sistem. & \cite{falisat-csf} \\ 
        \midrule
        \textbf{Teknologi} & TT1 & \raggedright Keamanan yang kuat & \raggedright Keamanan data adalah prioritas utama untuk melindungi informasi sensitif. & \cite{falisat-csf} \\ 
        \midrule
        \textbf{Teknologi} & TT2 & \raggedright Skalabilitas infrastruktur IT & \raggedright Infrastruktur IT harus dapat diskalakan untuk mendukung pertumbuhan masa depan. & \cite{Ghafari-csf}, \cite{salu-csf} \\ 
        \midrule
        \textbf{Teknologi} & TT3 & \raggedright Integrasi Sistem & \raggedright Sistem baru harus terintegrasi dengan lancar dengan sistem yang ada. & \cite{Ghafari-csf} \\ 
        \midrule
        \textbf{Teknologi} & TT4 & \raggedright Performa dan kecepatan sistem & \raggedright Performa sistem yang cepat meningkatkan efisiensi operasional dan kepuasan pengguna. & \cite{Ghafari-csf} \\ 
        \bottomrule
    \end{tabular}
\end{adjustwidth}
\end{table}


Setelah melakukan tinjaun pustaka, langkah-langkah dari \textit{Analytical Hierarchy Process} (AHP) diaplikasikan untuk merancang hierarki yang berisi tujuan, kriteria untuk mencapai tujuan, dan alternatif. AHP adalah metode sistematis untuk mengatasi tantangan dalam proses pengambilan keputusan dengan menggunakan prinsip-prinsip dari matematika. Metode ini menawarkan cara yang jelas dan terorganisasi untuk mengukur berbagai faktor yang terlibat dalam pengambilan keputusan dalam hierarki. Prosesnya dimulai dengan mengidentifikasi kriteria untuk keputusan tersebut \cite{Tavana2021AnalyticalHP}. Hasil dari penggunaan AHP adalah urutan faktor-faktor kesuksesan yang memengaruhi implementasi \textit{core banking system} pada lembaga keuangan mikro.

Penelitian ini menggunakan pendekatan kuantitatif dengan menggunakan data yang dikumpulkan dari kuesioner yang dirancang berdasarkan hierarki yang telah dibuat sebelumnya mengunakan langkah-langkah AHP. Jumlah responden yang mengisi kuesioner adalah 12 orang yang berasal dari 6 organisasi yang berbeda. Responden yang dipilih adalah 1 direktur atau ketua organisasi dan 1 ketua project dari setiap organisasi. Responden diminta untuk meninjau dan mengisi kuesioner yang berisi faktor-faktor keberhasilan implementasi \textit{core banking system}. Kemudian, responden memberikan nilai pada pertanyaan di kuesioner. 

Setelah responden mengisi kuesioner yang diberikan peneliti, urutan dari faktor-faktor kesuksesan dicari menggunakan langkah AHP selanjutnya yaitu menghitung rasio konsistensi atau \textit{consistency ratio} (CR). CSF harus diidentifikasi untuk mengurangi tingkat kegagalan implementasi \textit{core banking system}. Figure \ref{fig:tahapan-penelitian} di bawah merupakan ringkasan dari tahapan penelitian.

\begin{figure}[H]
\label{fig:tahapan-penelitian}
\includegraphics[width=6 cm]{attachments/tahapan.png}
\caption{Tahapan Penelitian}
\end{figure}   

\subsection{Analytical Hierarchy Process (AHP)}
Analytical Hierarchy Process (AHP) adalah sebuah teknik yang diperkenalkan oleh Dr. Thomas Saaty pada tahun 1990 melalui penelitiannya yang berjudul “How to make a decision: The analytic hierarchy process”. AHP digunakan untuk membantu organisasi membuat keputusan yang tepat melalui analisis solusi alternatif. 

Tahapan dari penggunaan teknik AHP adalah sebagai berikut:
\begin{enumerate}
    \item Mendefinisikan masalah dan kriteria yang ingin dievaluasi.
    \item Membuat struktur hirarki dengan tujuan di paling atas, kriteria dan sub-kriteria di tengah, serta alternatif di bagian bawah.
    \item Menentukan nilai penting dari setiap kriteria \cite{Jacob2021}.
\end{enumerate}

Teknik AHP diaplikasikan dengan menggunakan kuesioner matriks AHP berdasarkan faktor-faktor kesuksesan yang telah diidentifikasi sebelumnya. Terdapat skala angka dari 1 sampai 9 dengan keterangan sebagai berikut \cite{Saaty1990}:
\begin{itemize}
    \item 1: Kriteria yang sama pentingnya dengan kriteria lain.
    \item 3: Kriteria sedikit lebih penting dari yang lain.
    \item 5: Kriteria lebih penting dari yang lain.
    \item 7: Kriteria sangat penting dari yang lain.
    \item 9: Kriteria yang penting secara absolut dari yang lain.
    \item Nilai di antaranya merepresentasikan nilai tengah.
\end{itemize}

Menurut penelitian Singh \cite{Singh2019}, langkah-langkah teknik AHP adalah sebagai berikut:
\begin{enumerate}
    \item Menentukan tujuan dan kriteria untuk mencapainya.
    \item Mencari solusi alternatif untuk mencapai tujuan.
    \item Menentukan nilai setiap kriteria menggunakan matriks, untuk mencari nilai penting dari setiap kriteria.
    \item Membandingkan setiap pasangan kriteria menggunakan angka yang diperoleh dari matriks AHP, dengan memberikan urutan seberapa besar satu kriteria dibandingkan dengan yang lain.
    \item Menghitung prioritas relatif gabungan setiap kriteria untuk mengetahui nilai keseluruhan untuk setiap kriteria.
\end{enumerate}

Pada tahap perbandingan, penelitian ini menggunakan metode \textit{pairwise comparison}. Dalam proses ini, ditentukan seberapa penting setiap kriteria dibandingkan dengan kriteria lainnya dengan terlebih dahulu membuat matriks penilaian resiprokal $n \times n$ berdasarkan perbandingan kriteria. Matriks perbandingan (\textit{Matriks A}) dibuat menggunakan data yang dikumpulkan dari kuesioner dan wawancara dengan para ahli yang terlibat dalam proses pengambilan keputusan \cite{Singh2019}. Matriks perbandingan dapat dilihat pada persamaan \ref{eq:matriks_a}

\begin{equation}
A = \begin{bmatrix}
a_{11} & a_{12} & \cdots & a_{1m} \\
a_{21} & \cdots & \cdots & a_{2m} \\
\vdots & \vdots & \ddots & \vdots \\
a_{m1} & \cdots & \cdots & a_{mm}
\end{bmatrix} = 
\begin{bmatrix}
1 & a_{12} & \cdots & a_{1m} \\
\frac{1}{a_{12}} & 1 & \cdots & a_{2m} \\
\vdots & \vdots & \ddots & \vdots \\
\frac{1}{a_{1m}} & \cdots & \cdots & 1
\end{bmatrix}
\label{eq:matriks_a}
\end{equation}

Setelah pembuatan matriks \ref{eq:matriks_a}, dilakukan normalisasi terhadap matriks dengan rumus normalisasi pada persamaan \ref{eq:normalisasi} \cite{Singh2019}.

\begin{equation}
x_{ij} = \frac{a_{ij}}{\sum_{i=1}^{n} a_{ij}}
\label{eq:normalisasi}
\end{equation}

Kemudian dihitung bobot prioritas setiap pasangan kriteria dengan rumus pada persamaan \ref{eq:bobot_prioritas} \cite{Singh2019}:

\begin{equation}
P_j = \frac{1}{n} \sum_{i=1}^{n} a_{ij}
\label{eq:bobot_prioritas}
\end{equation}

Setelah itu, dilakukan pengecekan terhadap nilai relatif bobot menggunakan matriks perbandingan (A) dan eigenvector paling besar $\lambda_{\text{max}}$ dengan rumus pada persamaan \ref{eq:cek_nilai} \cite{Singh2019}:
\begin{equation}
AP_i = \lambda_{\text{max}} P_i \quad (i = 1, 2, 3, \ldots, n)
\label{eq:cek_nilai}
\end{equation}

Kemudian dilakukan kalkulasi untuk mencari \textit{Consistency Index} (CI) dengan rumus pada persamaan \ref{eq:ci} \cite{Singh2019}:
\begin{equation}
CI = \frac{\lambda_{\text{max}} - n}{n - 1}
\label{eq:ci}
\end{equation}

Rasio terakhir yang harus dihitung adalah \textit{Consistency Ratio} (CR). Supaya dianggap dapat diterima, nilai CR harus sekitar 0,10 (atau 10\%). Rumus untuk menghitung CR dapat dilihat pada persamaan \ref{eq:cr} \cite{Singh2019}:

\begin{equation}
CR = \frac{CI}{RI}
\label{eq:cr}
\end{equation}

Di mana RI adalah \textit{Random Index} yang biasanya diperoleh dari tabel yang telah ditetapkan berdasarkan jumlah perbandingan (n) yang dilakukan \cite{Singh2019}. RI dapat dilihat pada tabel \ref{random-index}.

\begin{table}[H]
    \caption{Random Index (RI)}
    \label{random-index}
    \centering
    \begin{tabular}{c|ccccccccccc}
        \toprule
        $n$ & 1 & 2 & 3 & 4 & 5 & 6 & 7 & 8 & 9 & 10 & 11 \\
        \midrule
        RI & 0 & 0 & 0.58 & 0.9 & 1.12 & 1.24 & 1.32 & 1.41 & 1.45 & 1.49 & 1.51 \\
        \bottomrule
    \end{tabular}
\end{table}



% https://remote-lib.ui.ac.id:2147/stamp/stamp.jsp?tp=&arnumber=8982404
% https://remote-lib.ui.ac.id:2147/stamp/stamp.jsp?tp=&arnumber=9354331



% Dibawah  ini template bawaan dari MDPI

% The introduction should briefly place the study in a broad context and highlight why it is important. It should define the purpose of the work and its significance. The current state of the research field should be reviewed carefully and key publications cited. Please highlight controversial and diverging hypotheses when necessary. Finally, briefly mention the main aim of the work and highlight the principal conclusions. As far as possible, please keep the introduction comprehensible to scientists outside your particular field of research. Citing a journal paper \cite{ref-journal}. Now citing a book reference \cite{ref-book1,ref-book2} or other reference types \cite{ref-unpublish,ref-communication,ref-proceeding}. Please use the command \citep{ref-thesis,ref-url} for the following MDPI journals, which use author--date citation: Administrative Sciences, Arts, Econometrics, Economies, Genealogy, Humanities, IJFS, Journal of Intelligence, Journalism and Media, JRFM, Languages, Laws, Religions, Risks, Social Sciences, Literature.
%%%%%%%%%%%%%%%%%%%%%%%%%%%%%%%%%%%%%%%%%%
% \section{Materials and Methods}

% Materials and Methods should be described with sufficient details to allow others to replicate and build on published results. Please note that publication of your manuscript implicates that you must make all materials, data, computer code, and protocols associated with the publication available to readers. Please disclose at the submission stage any restrictions on the availability of materials or information. New methods and protocols should be described in detail while well-established methods can be briefly described and appropriately cited.

% Research manuscripts reporting large datasets that are deposited in a publicly avail-able database should specify where the data have been deposited and provide the relevant accession numbers. If the accession numbers have not yet been obtained at the time of submission, please state that they will be provided during review. They must be provided prior to publication.

% Interventionary studies involving animals or humans, and other studies require ethical approval must list the authority that provided approval and the corresponding ethical approval code.
% \begin{quote}
% This is an example of a quote.
% \end{quote}

% %%%%%%%%%%%%%%%%%%%%%%%%%%%%%%%%%%%%%%%%%%
% \section{Results}

% This section may be divided by subheadings. It should provide a concise and precise description of the experimental results, their interpretation as well as the experimental conclusions that can be drawn.
% \subsection{Subsection}
% \subsubsection{Subsubsection}

% Bulleted lists look like this:
% \begin{itemize}
% \item	First bullet;
% \item	Second bullet;
% \item	Third bullet.
% \end{itemize}

% Numbered lists can be added as follows:
% \begin{enumerate}
% \item	First item; 
% \item	Second item;
% \item	Third item.
% \end{enumerate}

% The text continues here. 

% \subsection{Figures, Tables and Schemes}

% All figures and tables should be cited in the main text as Figure~\ref{fig1}, Table~\ref{tab1}, etc.

% \begin{figure}[H]
% \includegraphics[width=10.5 cm]{Definitions/logo-mdpi}
% \caption{This is a figure. Schemes follow the same formatting. If there are multiple panels, they should be listed as: (\textbf{a}) Description of what is contained in the first panel. (\textbf{b}) Description of what is contained in the second panel. Figures should be placed in the main text near to the first time they are cited. A caption on a single line should be centered.\label{fig1}}
% \end{figure}   
% \unskip

% \begin{table}[H] 
% \caption{This is a table caption. Tables should be placed in the main text near to the first time they are~cited.\label{tab1}}
% %\newcolumntype{C}{>{\centering\arraybackslash}X}
% \begin{tabularx}{\textwidth}{CCC}
% \toprule
% \textbf{Title 1}	& \textbf{Title 2}	& \textbf{Title 3}\\
% \midrule
% Entry 1		& Data			& Data\\
% Entry 2		& Data			& Data \textsuperscript{1}\\
% \bottomrule
% \end{tabularx}
% \noindent{\footnotesize{\textsuperscript{1} Tables may have a footer.}}
% \end{table}

% The text continues here (Figure~\ref{fig2} and Table~\ref{tab2}).

% % Example of a figure that spans the whole page width. The same concept works for tables, too.
% \begin{figure}[H]
% \begin{adjustwidth}{-\extralength}{0cm}
% \centering
% \includegraphics[width=15.5cm]{Definitions/logo-mdpi}
% \end{adjustwidth}
% \caption{This is a wide figure.\label{fig2}}
% \end{figure}  

% \begin{table}[H]
% \caption{This is a wide table.\label{tab2}}
% 	\begin{adjustwidth}{-\extralength}{0cm}
% %		\newcolumntype{C}{>{\centering\arraybackslash}X}
% 		\begin{tabularx}{\fulllength}{CCCC}
% 			\toprule
% 			\textbf{Title 1}	& \textbf{Title 2}	& \textbf{Title 3}     & \textbf{Title 4}\\
% 			\midrule
% \multirow[m]{3}{*}{Entry 1 *}	& Data			& Data			& Data\\
% 			  	                   & Data			& Data			& Data\\
% 			             	      & Data			& Data			& Data\\
%                     \midrule
% \multirow[m]{3}{*}{Entry 2}    & Data			& Data			& Data\\
% 			  	                  & Data			& Data			& Data\\
% 			             	     & Data			& Data			& Data\\
%                     \midrule
% \multirow[m]{3}{*}{Entry 3}    & Data			& Data			& Data\\
% 			  	                 & Data			& Data			& Data\\
% 			             	    & Data			& Data			& Data\\
%                   \midrule
% \multirow[m]{3}{*}{Entry 4}   & Data			& Data			& Data\\
% 			  	                 & Data			& Data			& Data\\
% 			             	    & Data			& Data			& Data\\
% 			\bottomrule
% 		\end{tabularx}
% 	\end{adjustwidth}
% 	\noindent{\footnotesize{* Tables may have a footer.}}
% \end{table}

% %\begin{listing}[H]
% %\caption{Title of the listing}
% %\rule{\columnwidth}{1pt}
% %\raggedright Text of the listing. In font size footnotesize, small, or normalsize. Preferred format: left aligned and single spaced. Preferred border format: top border line and bottom border line.
% %\rule{\columnwidth}{1pt}
% %\end{listing}

% Text.

% Text.

% \subsection{Formatting of Mathematical Components}

% This is the example 1 of equation:
% \begin{linenomath}
% \begin{equation}
% a = 1,
% \end{equation}
% \end{linenomath}
% the text following an equation need not be a new paragraph. Please punctuate equations as regular text.
% %% If the documentclass option "submit" is chosen, please insert a blank line before and after any math environment (equation and eqnarray environments). This ensures correct linenumbering. The blank line should be removed when the documentclass option is changed to "accept" because the text following an equation should not be a new paragraph.

% This is the example 2 of equation:
% \begin{adjustwidth}{-\extralength}{0cm}
% \begin{equation}
% a = b + c + d + e + f + g + h + i + j + k + l + m + n + o + p + q + r + s + t + u + v + w + x + y + z
% \end{equation}
% \end{adjustwidth}

% % Example of a page in landscape format (with table and table footnote).
% %\startlandscape
% %\begin{table}[H] %% Table in wide page
% %\caption{This is a very wide table.\label{tab3}}
% %	\begin{tabularx}{\textwidth}{CCCC}
% %		\toprule
% %		\textbf{Title 1}	& \textbf{Title 2}	& \textbf{Title 3}	& \textbf{Title 4}\\
% %		\midrule
% %		Entry 1		& Data			& Data			& This cell has some longer content that runs over two lines.\\
% %		Entry 2		& Data			& Data			& Data\textsuperscript{1}\\
% %		\bottomrule
% %	\end{tabularx}
% %	\begin{adjustwidth}{+\extralength}{0cm}
% %		\noindent\footnotesize{\textsuperscript{1} This is a table footnote.}
% %	\end{adjustwidth}
% %\end{table}
% %\finishlandscape


% Please punctuate equations as regular text. Theorem-type environments (including propositions, lemmas, corollaries etc.) can be formatted as follows:
% %% Example of a theorem:
% \begin{Theorem}
% Example text of a theorem.
% \end{Theorem}

% The text continues here. Proofs must be formatted as follows:

% %% Example of a proof:
% \begin{proof}[Proof of Theorem 1]
% Text of the proof. Note that the phrase ``of Theorem 1'' is optional if it is clear which theorem is being referred to.
% \end{proof}
% The text continues here.

% %%%%%%%%%%%%%%%%%%%%%%%%%%%%%%%%%%%%%%%%%%
% \section{Discussion}

% Authors should discuss the results and how they can be interpreted from the perspective of previous studies and of the working hypotheses. The findings and their implications should be discussed in the broadest context possible. Future research directions may also be highlighted.

% %%%%%%%%%%%%%%%%%%%%%%%%%%%%%%%%%%%%%%%%%%
% \section{Conclusions}

% This section is not mandatory, but can be added to the manuscript if the discussion is unusually long or complex.

% %%%%%%%%%%%%%%%%%%%%%%%%%%%%%%%%%%%%%%%%%%
% \section{Patents}

% This section is not mandatory, but may be added if there are patents resulting from the work reported in this manuscript.

%%%%%%%%%%%%%%%%%%%%%%%%%%%%%%%%%%%%%%%%%%
\vspace{6pt} 

%%%%%%%%%%%%%%%%%%%%%%%%%%%%%%%%%%%%%%%%%%
%% optional
%\supplementary{The following supporting information can be downloaded at:  \linksupplementary{s1}, Figure S1: title; Table S1: title; Video S1: title.}

% Only for journal Methods and Protocols:
% If you wish to submit a video article, please do so with any other supplementary material.
% \supplementary{The following supporting information can be downloaded at: \linksupplementary{s1}, Figure S1: title; Table S1: title; Video S1: title. A supporting video article is available at doi: link.}

% Only for journal Hardware:
% If you wish to submit a video article, please do so with any other supplementary material.
% \supplementary{The following supporting information can be downloaded at: \linksupplementary{s1}, Figure S1: title; Table S1: title; Video S1: title.\vspace{6pt}\\
%\begin{tabularx}{\textwidth}{lll}
%\toprule
%\textbf{Name} & \textbf{Type} & \textbf{Description} \\
%\midrule
%S1 & Python script (.py) & Script of python source code used in XX \\
%S2 & Text (.txt) & Script of modelling code used to make Figure X \\
%S3 & Text (.txt) & Raw data from experiment X \\
%S4 & Video (.mp4) & Video demonstrating the hardware in use \\
%... & ... & ... \\
%\bottomrule
%\end{tabularx}
%}

%%%%%%%%%%%%%%%%%%%%%%%%%%%%%%%%%%%%%%%%%%
% \authorcontributions{For research articles with several authors, a short paragraph specifying their individual contributions must be provided. The following statements should be used ``Conceptualization, X.X. and Y.Y.; methodology, X.X.; software, X.X.; validation, X.X., Y.Y. and Z.Z.; formal analysis, X.X.; investigation, X.X.; resources, X.X.; data curation, X.X.; writing---original draft preparation, X.X.; writing---review and editing, X.X.; visualization, X.X.; supervision, X.X.; project administration, X.X.; funding acquisition, Y.Y. All authors have read and agreed to the published version of the manuscript.'', please turn to the  \href{http://img.mdpi.org/data/contributor-role-instruction.pdf}{CRediT taxonomy} for the term explanation. Authorship must be limited to those who have contributed substantially to the work~reported.}

% \funding{Please add: ``This research received no external funding'' or ``This research was funded by NAME OF FUNDER grant number XXX.'' and  and ``The APC was funded by XXX''. Check carefully that the details given are accurate and use the standard spelling of funding agency names at \url{https://search.crossref.org/funding}, any errors may affect your future funding.}

% \institutionalreview{In this section, you should add the Institutional Review Board Statement and approval number, if relevant to your study. You might choose to exclude this statement if the study did not require ethical approval. Please note that the Editorial Office might ask you for further information. Please add “The study was conducted in accordance with the Declaration of Helsinki, and approved by the Institutional Review Board (or Ethics Committee) of NAME OF INSTITUTE (protocol code XXX and date of approval).” for studies involving humans. OR “The animal study protocol was approved by the Institutional Review Board (or Ethics Committee) of NAME OF INSTITUTE (protocol code XXX and date of approval).” for studies involving animals. OR “Ethical review and approval were waived for this study due to REASON (please provide a detailed justification).” OR “Not applicable” for studies not involving humans or animals.}

% \informedconsent{Any research article describing a study involving humans should contain this statement. Please add ``Informed consent was obtained from all subjects involved in the study.'' OR ``Patient consent was waived due to REASON (please provide a detailed justification).'' OR ``Not applicable'' for studies not involving humans. You might also choose to exclude this statement if the study did not involve humans.

% Written informed consent for publication must be obtained from participating patients who can be identified (including by the patients themselves). Please state ``Written informed consent has been obtained from the patient(s) to publish this paper'' if applicable.}

% \dataavailability{We encourage all authors of articles published in MDPI journals to share their research data. In this section, please provide details regarding where data supporting reported results can be found, including links to publicly archived datasets analyzed or generated during the study. Where no new data were created, or where data is unavailable due to privacy or ethical restrictions, a statement is still required. Suggested Data Availability Statements are available in section ``MDPI Research Data Policies'' at \url{https://www.mdpi.com/ethics}.} 

% Only for journal Nursing Reports
%\publicinvolvement{Please describe how the public (patients, consumers, carers) were involved in the research. Consider reporting against the GRIPP2 (Guidance for Reporting Involvement of Patients and the Public) checklist. If the public were not involved in any aspect of the research add: ``No public involvement in any aspect of this research''.}

% Only for journal Nursing Reports
%\guidelinesstandards{Please add a statement indicating which reporting guideline was used when drafting the report. For example, ``This manuscript was drafted against the XXX (the full name of reporting guidelines and citation) for XXX (type of research) research''. A complete list of reporting guidelines can be accessed via the equator network: \url{https://www.equator-network.org/}.}

% Only for journal Nursing Reports
%\useofartificialintelligence{Please describe in detail any and all uses of artificial intelligence (AI) or AI-assisted tools used in the preparation of the manuscript. This may include, but is not limited to, language translation, language editing and grammar, or generating text. Alternatively, please state that “AI or AI-assisted tools were not used in drafting any aspect of this manuscript”.}

% \acknowledgments{In this section you can acknowledge any support given which is not covered by the author contribution or funding sections. This may include administrative and technical support, or donations in kind (e.g., materials used for experiments).}

% \conflictsofinterest{Declare conflicts of interest or state ``The authors declare no conflicts of interest.'' Authors must identify and declare any personal circumstances or interest that may be perceived as inappropriately influencing the representation or interpretation of reported research results. Any role of the funders in the design of the study; in the collection, analyses or interpretation of data; in the writing of the manuscript; or in the decision to publish the results must be declared in this section. If there is no role, please state ``The funders had no role in the design of the study; in the collection, analyses, or interpretation of data; in the writing of the manuscript; or in the decision to publish the results''.} 

%%%%%%%%%%%%%%%%%%%%%%%%%%%%%%%%%%%%%%%%%%
%% Optional

%% Only for journal Encyclopedia
%\entrylink{The Link to this entry published on the encyclopedia platform.}

\abbreviations{Abbreviations}{
The following abbreviations are used in this manuscript:\\

\noindent 
\begin{tabular}{@{}ll}
MDPI & Multidisciplinary Digital Publishing Institute\\
DOAJ & Directory of open access journals\\
TLA & Three letter acronym\\
LD & Linear dichroism
\end{tabular}
}

%%%%%%%%%%%%%%%%%%%%%%%%%%%%%%%%%%%%%%%%%%
%% Optional
\appendixtitles{no} % Leave argument "no" if all appendix headings stay EMPTY (then no dot is printed after "Appendix A"). If the appendix sections contain a heading then change the argument to "yes".
\appendixstart
\appendix
\section[\appendixname~\thesection]{}
\subsection[\appendixname~\thesubsection]{}
The appendix is an optional section that can contain details and data supplemental to the main text---for example, explanations of experimental details that would disrupt the flow of the main text but nonetheless remain crucial to understanding and reproducing the research shown; figures of replicates for experiments of which representative data are shown in the main text can be added here if brief, or as Supplementary Data. Mathematical proofs of results not central to the paper can be added as an appendix.

\begin{table}[H] 
\caption{This is a table caption.\label{tab5}}
\newcolumntype{C}{>{\centering\arraybackslash}X}
\begin{tabularx}{\textwidth}{CCC}
\toprule
\textbf{Title 1}	& \textbf{Title 2}	& \textbf{Title 3}\\
\midrule
Entry 1		& Data			& Data\\
Entry 2		& Data			& Data\\
\bottomrule
\end{tabularx}
\end{table}

\section[\appendixname~\thesection]{}
All appendix sections must be cited in the main text. In the appendices, Figures, Tables, etc. should be labeled, starting with ``A''---e.g., Figure A1, Figure A2, etc.

%%%%%%%%%%%%%%%%%%%%%%%%%%%%%%%%%%%%%%%%%%
\begin{adjustwidth}{-\extralength}{0cm}
%\printendnotes[custom] % Un-comment to print a list of endnotes

\reftitle{References}

% Please provide either the correct journal abbreviation (e.g. according to the “List of Title Word Abbreviations” http://www.issn.org/services/online-services/access-to-the-ltwa/) or the full name of the journal.
% Citations and References in Supplementary files are permitted provided that they also appear in the reference list here. 

%=====================================
% References, variant A: external bibliography
%=====================================
\bibliography{refs}

%=====================================
% References, variant B: internal bibliography
%=====================================
% \begin{thebibliography}{999}
% % Reference 1
% \bibitem[Author1(year)]{ref-journal}
% Author~1, T. The title of the cited article. {\em Journal Abbreviation} {\bf 2008}, {\em 10}, 142--149.

% % Reference 2
% \bibitem[Author2(year)]{ref-book1}
% Author~2, L. The title of the cited contribution. In {\em The Book Title}; Editor 1, F., Editor 2, A., Eds.; Publishing House: City, Country, 2007; pp. 32--58.
% % Reference 3
% \bibitem[Author3(year)]{ref-book2}
% Author 1, A.; Author 2, B. \textit{Book Title}, 3rd ed.; Publisher: Publisher Location, Country, 2008; pp. 154--196.
% % Reference 4
% \bibitem[Author4(year)]{ref-unpublish}
% Author 1, A.B.; Author 2, C. Title of Unpublished Work. \textit{Abbreviated Journal Name} year, \textit{phrase indicating stage of publication (submitted; accepted; in press)}.
% % Reference 5
% \bibitem[Author5(year)]{ref-communication}
% Author 1, A.B. (University, City, State, Country); Author 2, C. (Institute, City, State, Country). Personal communication, 2012.
% % Reference 6
% \bibitem[Author6(year)]{ref-proceeding}
% Author 1, A.B.; Author 2, C.D.; Author 3, E.F. Title of presentation. In Proceedings of the Name of the Conference, Location of Conference, Country, Date of Conference (Day Month Year); Abstract Number (optional), Pagination (optional).
% % Reference 7
% \bibitem[Author7(year)]{ref-thesis}
% Author 1, A.B. Title of Thesis. Level of Thesis, Degree-Granting University, Location of University, Date of Completion.
% % Reference 8
% \bibitem[Author8(year)]{ref-url}
% Title of Site. Available online: URL (accessed on Day Month Year).
% \end{thebibliography}

% If authors have biography, please use the format below
%\section*{Short Biography of Authors}
%\bio
%{\raisebox{-0.35cm}{\includegraphics[width=3.5cm,height=5.3cm,clip,keepaspectratio]{Definitions/author1.pdf}}}
%{\textbf{Firstname Lastname} Biography of first author}
%
%\bio
%{\raisebox{-0.35cm}{\includegraphics[width=3.5cm,height=5.3cm,clip,keepaspectratio]{Definitions/author2.jpg}}}
%{\textbf{Firstname Lastname} Biography of second author}

% For the MDPI journals use author-date citation, please follow the formatting guidelines on http://www.mdpi.com/authors/references
% To cite two works by the same author: \citeauthor{ref-journal-1a} (\citeyear{ref-journal-1a}, \citeyear{ref-journal-1b}). This produces: Whittaker (1967, 1975)
% To cite two works by the same author with specific pages: \citeauthor{ref-journal-3a} (\citeyear{ref-journal-3a}, p. 328; \citeyear{ref-journal-3b}, p.475). This produces: Wong (1999, p. 328; 2000, p. 475)

%%%%%%%%%%%%%%%%%%%%%%%%%%%%%%%%%%%%%%%%%%
%% for journal Sci
%\reviewreports{\\
%Reviewer 1 comments and authors’ response\\
%Reviewer 2 comments and authors’ response\\
%Reviewer 3 comments and authors’ response
%}
%%%%%%%%%%%%%%%%%%%%%%%%%%%%%%%%%%%%%%%%%%
\PublishersNote{}
\end{adjustwidth}
\end{document}

